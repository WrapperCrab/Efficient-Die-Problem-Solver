\documentclass{article}
\usepackage{amssymb}
\usepackage{amsmath}

\title{Efficient Die Problem Algorithm}
\author{Winter Cross}
\date{June 2024}

\begin{document}

\maketitle

\section{The Goal of this Algorithm}
All information in this section is also explained in the README.

\subsection{Any Die can Simulate Any Rational Probability}
Imagine you are playing an rpg where you want a player to have a $1$ in $5$ chance of success and a $4$ in $5$ chance of failure.
Easy enough if you have a $5$-sided die lying around, but unfortunately, you only have a coin!\newline
Luckily, it's still possible and not even that hard.\newline

What you do is flip the coin $3$ times, keeping track of the results each time.\newline
Let's say that tails is $0$ and heads is $1$.\newline
From our flip, we can create a $3$-digit binary number. Suppose we rolled $101$ which is $5$ in decimal.\newline
Then, we simply follow this algorithm:\newline

\noindent
if $1=result$, then it's a success!\newline
if $1<result\leq5$, then it's a failure.\newline
if $5<result\leq8$, then we must retry the simulation.\newline

And this will simulate a probability of $1/5$ with a coin. The average number of rolls this will take is $4.8$.\newline

This idea can be generalized to any die with at least $2$ sides, and any rational probability.

\subsection{What is there to Solve?}
Consider the example from the previous section.\newline
What if instead of thinking of the probability as 1/5, we thought of it as 3/15?\newline

Then, we could flip our coin 4 times to get a number 1 to 16.\newline

\noindent
if $1\leq result\leq 3$, then it's a success!\newline
if $3<result\leq 15$, then it's a failure.\newline
if $result=16$, then we must retry the simulation.\newline

This also simulates a probability of 1/5 with a coin. However, the probability to retry is much lower, and as such, the average number of needed rolls is only 4.26667 which is smaller than the previous strategy.\newline

So, this script finds the most efficient way to simulate a given probability with a given die.\newline

\noindent
It optimizes for: (ties are broken by following list item)\newline
\begin{enumerate}
    \item Lowest Expected Number of Rolls
    \item Lowest Probability for Retry
    \item Lowest Number of Rolls per Attempt
\end{enumerate}

\section{Finding the Algorithm}
This section goes through a derivation of the algorithm used in this script.

\subsection{Variables}
We'll start by defining some of the variables we will be using:\newline

\noindent
$n=$ number of sides on the die \newline
$a/b=$ the probability to simulate as a fraction of two integers\newline
$r=$ the number of rolls used in one attempt of a simulation\newline
$p=$ the probability that a simulation attempt ends in success or failure\newline
$q=1-p=$ the probability of needing to retry after one simulation attempt\newline
$E=$ the expected number of rolls for a strategy\newline
$v=$ the scaling factor used for a strategy

\subsection{Categorizing All Strategies}
In order to find the best strategy, we need to understand all of the strategies that are possible.
In the example in the README, we analyzed two strategies for simulating a probability of $1/5$ with a coin ($n=2$).

We found that viewing $1/5$ as $3/15$ created a new strategy that was actually more efficient than the first.

Given any unsimplified fraction $a/b$, a die with $n$ sides, and a satisfactory number of rolls, $r$, we can create a unique strategy like so:\newline

\noindent
Roll the die $r$ times to get an $r$-digit number in base $n$.\newline
If $1\leq result \leq a$, it's a success\newline
If $a<result \leq b$, it's a failure\newline
If $b< result \leq n^r$, you must retry the strategy\newline

Not all values of $r$ work. We need at least as many outcomes as the denominator $b$, so we must find a value $r$ such that rolling $n$ $r$ times results in a value $\geq b$. In math,

$$n^r\geq b$$
$$r\geq \log_n b$$
$$r\geq\lceil\log_n b\rceil$$

But not all of these $r$-values give worthwhile strategies. Consider the simple example $n=2, a=1, b=3, r=2$.\newline
If $1\leq result \leq 1$, it's a success\newline
If $1<result \leq 3$, it's a failure\newline
If $3< result \leq 4$, you must retry the strategy\newline

This strategy has a chance for failure of $q=0.25$.

Now, consider if we changed nothing but increased the number of rolls to $3$. Then we have \newline
If $1\leq result \leq 1$, it's a success\newline
If $1<result \leq 3$, it's a failure\newline
If $3< result \leq 8$, you must retry the strategy\newline

Which is very similar, but has a huge failure rate of $1/2$.

From this, hopefully it is obvious that it is always in your best interest to take the lowest $r$ value as possible. Thus, we can state

$$r=r(a,b,n)=\lceil\log_n b\rceil$$

So, we can create a unique strategy by taking the values $a, b,$ and $n$. Now, our goal is to find the best of the strategies where $a/b$ is equal to our wanted probability.

\subsection{The Role of the Scaling Factor}
All fractions that are equal to $a/b$ can be described with a scaling factor $v$ so that our new fraction is $va/vb$. The problem we are trying to solve is, given a probability $a/b$ where $\gcd(a,b)=1$ and a die of $n$ sides, what value $v$ gives the strategy with the lowest expected number of rolls?

\subsection{Strategy Formulae}
given $n, a,$ and $b$, and calculating $r$ as given above, we can calculate the expected number of rolls for the resulting strategy.\newline

The probability to finish a strategy on one attempt is $p=\frac{b}{n^r}$. \newline

The probability of retrying is $q=1-p$ \newline

Finishing on the first attempt results in $r$ rolls, $2r$ after 2, $3r$ after 3 and so on. Thus, we arrive at the sum

$$E=E(a,b,n) = pr+pq2r+pq^23r+pq^34r+...$$
$$= \sum_{i=0}^{i=\infty}pq^i(i+1)r$$
$$= pr\sum_{i=0}^{i=\infty}q^i(i+1)$$
$$=pr\frac{1}{p^2}$$
$$=\frac{r}{p} = \frac{rn^r}{b}$$

When we apply a scaling factor of $v$, the expected number of rolls is

$$E(av,bv,n) = \frac{r_vn^{r_v}}{bv}$$
where
$$r_v=r(av,bv,n)=\lceil\log_n bv\rceil$$

\subsection{Limits on $v$}
There are infinity possible values of $v$, so we can't possibly check them all. Instead, we must place some bounds on $v$ to limit our search.\newline

There are 2 useful limits we can put on $v$ that will make our search much shorter

\subsubsection{Limit 1: A $v$ for each $r$}
Suppose we have two scaling factors $v_1<v_2$ such that 
$$r(av_1,bv_1,n)=r(av_2,bv_2,n)=r$$
This is a common scenario, for example $a=1,b=2,n=6$ with $v_1=1$ and $v_2=2$ both only require $1$ roll of the die.\newline

Calculating our expected number of rolls for both scaling factors, we get
$$E_1=E(av_1,bv_1,n)=\frac{rn^{r}}{bv_1}$$
$$E_2=E(av_2,bv_2,n)=\frac{rn^{r}}{bv_2}$$
Since $v_1<v_2$, we get that $E_2<E_1$. Thus, we only care about one $v$-value per $r$-value, that being the largest $v$ that uses $r$ rolls.\newline

An equation for the largest $v$-value is $v=\lfloor\frac{n^r}{b}\rfloor$\newline

This means, that it we can iterate over $r$-values instead of $v$-values. But there are still infinity possible $r$'s, so we need another limit on our search.

\subsubsection{Limit 2: An Upper Bound}
Given that we can iterate over $r$, the first strategy we'll be interested in is the best strategy for the lowest possible value of $r$.\newline

The lowest value of $r$ is $r_1=r(a,b,n)$, so the first value of $v$ to check is $v_1=\lfloor\frac{n^{r_1}}{b}\rfloor$. Then, our first Expected value is $E_1=E(av_1,bv_1,n)$.\newline

$E_1$ may or may not be the best that we can do. But, consider the strategy that takes $r$ rolls where $r>E_1$. Then, even the first attempt alone uses more than $E_1$ rolls, meaning all strategies that take $r$ or more rolls are worse than the strategy we've already found!\newline

We only need to check the values of $r$ such that $r\leq \lfloor E_1\rfloor$. The list of $v$ values we must check looks something like this:

$$v=\lceil \frac{n^{r_1}}{b} \rceil,\lceil \frac{n^{r_1+1}}{b} \rceil, \lceil \frac{n^{r_1+2}}{b} \rceil, ... , \lceil \frac{n^{\lfloor E_1 \rfloor}}{b} \rceil $$

At last we have a finite list of items to check!

\subsection{The Final Algorithm}
So, our final algorithm starts by calculating $r_1, v_1,$ and $E_1$. Then, it iterates over $r$ to check the expected number of rolls for each strategy. If a more efficient strategy is found, the values are saved, and the upper bound on $r$ is lowered by the new $E$-value. Once the upper bound is reached, we know that we've found the best strategy and print the result in a readable format to the user. 



\end{document}
